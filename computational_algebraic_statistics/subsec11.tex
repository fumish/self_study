
\begin{frame}{線形な連立方程式}
以下の問題をまず考える
\begin{equation}
\left\{
   \begin{aligned}
   & x+2y-z=2 \\
   & x+y-4z=3 \\
   & x+3y+3z=0
   \end{aligned}
\right.
\end{equation}
これは、連立1次方程式なので、線形代数で学ぶ手法を使えば良い
\end{frame}

\begin{frame}
次に、以下の問題も考えてみる
\begin{equation}
\left\{
   \begin{aligned}
   & x^2+y^2+4z^2=81 \\
   & x-y+z^2=13 \\
   & xz-2y=18
   \end{aligned}
\right.
\end{equation}
これは、第2式の2倍を第3式から引いてyを消去し、$xz-2x-2z^2=-8$を得て、この因数分解$(z-2)(x-2z-4)=0$を行うことで、計算ができる。
\end{frame}

\begin{frame}
次の問題は少し難しい
\begin{equation}
\left\{
   \begin{aligned}
   & x^2+y^2+4z^2=90 \\
   & x-y+z=12 \\
   & xz-3y=28
   \end{aligned}
\right.
\end{equation}
これは、同じような因数分解ができない。
グレブナー基底の計算のような方法で解いてみようと思ったが、教科書と同じ式導出になりそうなので、教科書で説明
\end{frame}

\begin{frame} {注意点}

\begin{itemize}
\item 先ほどの手順はグレブナー基底の計算アルゴリズムになっていることが、後の章で分かる。
\item グレブナー基底は、各単項式(定義はいずれ出てくる)の順序を与えると得られるもので、今回の手順は、$x \succ y \succ z$で純辞書式順序という順序を用いた場合の結果になっている
\item 最終的に得られた連立方程式の形式は$z$だけの式だった。これは偶然ではなく、消去定理の結果から、純辞書式順序を用いると、連立方程式のうちの一つが一つの変数の方程式になるためである(これはすごいと思った)。
\end{itemize}
\end{frame}

\begin{frame} {実演}
ここまでの式のグレブナー基底をPython求めてみる: \\
Pythonでグレブナー基底を使う方法に例えばsympyがあるので、それを簡単に紹介する
\end{frame}
