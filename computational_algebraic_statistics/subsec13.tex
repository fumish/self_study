\begin{frame} {単項式イデアル}
	前節で、1変数の多項式環におけるイデアル記述問題とイデアル所属問題について考えた。
	次に多変数の多項式環について考えたいが、任意のイデアルを考えることは難しいので、
	単項式から生成されるイデアルである単項式イデアルについてまず考える。
	\begin{definition} {単項式イデアル}
		単項式からなる生成系を持つイデアルのことを、単項式イデアルと呼ぶ。
	\end{definition}
	例えば、$\langle x, xy, x^2 \rangle \subset K[x,y]$は単項式イデアルである。
	また、単項式イデアルの生成系は、必ず単項式の集合になるわけではなく、
	$\langle x_1^2, x_2^3 \rangle = \langle x_1^2, x_1^2+x_2^3 \rangle$などと表すことができる。
\end{frame}

\begin{frame}
	単項式を考えるので、単項式の集合を定義しておく。変数$x_1, \ldots, x_n$の単項式全体の集合を
	\begin{align*}
		M_n := \{ \prod_{i=1}^n x_i^{a_i} | a_i \in \mathbb{Z}_{\geq 0} \}
	\end{align*}
	と表す。
	このとき、単項式イデアルは、$M_n$の必ずしも有限でない部分集合$M \subset M_n$をもとに、
	$I = \langle \{ u | u \in M \} \rangle$と書くことができる。
\end{frame}

\begin{frame} {単項式イデアルの元となる単項式}
	有限生成な単項式イデアルの一例として、$I = \langle x_1 x_2^5, x_1^4 x_2^3, x_1^6 \rangle \subset K[x_1, x_2]$を考える。
	これは、$M_2$の有限な部分集合$\{x_1 x_2^5, x_1^4 x_2^3, x_1^6\}$が$I$の生成系となっていることが分かる。
	また、生成系の定義から、このイデアルに属する多項式$f$は、
	\begin{align*}
		f = h_1 x_1 x_2^5 + h_2 x_1^4 x_2^3 + h_3 x_1^6, h_1, h_2, h_3 in K[x_1, x_2]
	\end{align*}
	と書くことができるし、このように書ける元を集めたものが$I$でもある。\\
	$I$の特徴づけをする。生成系の元の単項式のいずれかで割り切れる単項式は、全て$I$の元であることがわかる。\\
	\begin{exampleblock} {例}
		$x_1 x_2^7 = x_2^2 (x_1 x_2^5) \in I$, であり$x_1 x_2^5$で割り切れる。
	\end{exampleblock}
\end{frame}

\begin{frame}
	逆に、ある\red{単項式$u$} $\in M_2$があって、これが$I$の元であるとする。
	このとき
	\begin{align*}
		u = h_1 x_1 x_2^5 + h_2 x_1^4 x_2^3 + h_3 x_1^6, h_1, h_2, h_3 \in K[x_1, x_2]
	\end{align*}
	と表せるが、左辺は単項式なので、右辺は$\{ x_1 x_2^5, x_1^4 x_2^3, x_1^6 \}$のいずれかで割り切れる。
	\begin{block} {コメント}
		今回の例では、$u$は3つの単項式のいずれかで割り切れるが、$h_i$は任意の多項式なので、基本的に項の数は3つ以上ある。
		ただ、それらの項は3つの単項式のいずれかで括れる(=割り切れる)ということを述べている。
	\end{block}
\end{frame}

\begin{frame}
	以上の議論を、必ずしも有限とは限らない単項式の集合$\{ u_{\lambda} | \lambda \in \Lambda \} \subset M_n$
	で生成される一般の単項式イデアルに拡張する。単項式イデアル$\langle \{ u_{\lambda} | \lambda \in \Lambda \} \rangle$
	の元である\red{単項式$u$}は
	\begin{align*}
		u = \sum_{\lambda \in \Lambda} g_{\lambda} u_{\lambda}, g_{\lambda} \in K[x_1, \ldots, x_n]
	\end{align*}
	と表すことができ、この右辺を展開して得られる項のうち係数が0でないものはただ一つであり、その項は$\{ u_{\lambda} | \lambda \in \Lambda \}$のいずれかの元で割り切れる。
	これを補題としてまとめると、以下の通りになる。
	\begin{lemma}
		\label{lem:single_monomial}
		$I = \langle \{u | u \in M\} \rangle$を$K[x_1, \ldots, x_n]$の単項式イデアルとすると、
		単項式$v \in M_n$が$I$の元であるための必要十分条件は、ある単項式$u \in M$が$v$を割り切ることである。
	\end{lemma}
	ここで単項式$u=x_1^{a_1} \cdots x_n^{a_n}$が単項式$v=x_1^{b_1} \cdots x_n^{b_n}$を割り切るとは、
	$a_i \leq b_i$が全ての$i$に対して成り立つことをいう。

\end{frame}

\begin{frame}
	なお、2変数の単項式イデアルと、それに属する単項式かどうかは
	$x^a y^b$を二次元平面上の点$(a, b)$と対応させることで確認できる(教科書の図1.1参照)。\\
	図1.1の各点は単項式イデアル$I = \langle xy^5, x^4y^3, x^6 \rangle$に含まれる多項式を表しているが、
	$(1,5), (4,3), (6,0)$を左下の点として、
	幅、高さが無限の四角形の和集合が$I$に含まれる単項式であることを表している(単項式の次数は非負整数なので、非負整数上のみ考える)。
\end{frame}

\begin{frame} {単項式イデアルの元となる多項式}
	次に単項式イデアル$I=\langle \{ u_{\lambda} | \lambda \in \Lambda \} \rangle$
	の元となる多項式$f \in I$を特徴づける。\\
	$f$を
	\begin{align*}
		f = \sum_{\lambda \in \Lambda} g_{\lambda} u_{\lambda}, g_{\lambda} \in K[x_1, \ldots, x_n]
	\end{align*}
	と表し、右辺に含まれる単項式を考える。それらはすべて、いずれかの$u_{\lambda}$で割り切れて、$I$の元の単項式であることがわかる。\\
	逆に、単項式イデアル$I$の元である単項式を任意に選び、それらの線形結合として、多項式$f$を作ればイデアルの性質から$f \in I$である。\\

\end{frame}

\begin{frame}
	これらをまとめて、補題にすると以下のようになる。
	\begin{lemma}
		\label{lem:monomial_ideal}
		$I$を$K[x_1, \ldots, x_n]$の単項式イデアルとし、$f \in K[x_1, \ldots, x_n]$とする。\\
		このとき、次の3つは互いに同値である。\\
		\begin{enumerate}
			\item $f \in I$
			\item $f$の全ての項は$I$に属する
			\item $f$は$I$の元である単項式の線形結合として表せる(単項式のイデアルの定義から成り立つような...)
		\end{enumerate}
	\end{lemma}
\end{frame}

\begin{frame} {単項式イデアルのイデアル所属問題}
	補題\ref{lem:single_monomial}, 補題\ref{lem:monomial_ideal}から、
	単項式イデアルのイデアル所属問題(ある多項式がイデアルの所属するか)の解を考える。\\
	$f \in K[x_1, \ldots, x_n]$が
	単項式イデアル$I = \langle \{ u |u \in M_n \} \rangle$に含まれるかどうかは、
	$f$の全ての単項式が$I$に含まれるかどうかを調べればよい(補題\ref{lem:monomial_ideal}の条件2より)。\\
	$f$の各単項式が$M_n$のいずれかの元で割り切れれば、補題\ref{lem:single_monomial}から
	その単項式は$I$に含まれるので$f$も$I$に含まれる。以上をまとめると、\\
	\begin{block} {単項式イデアルの所属問題}
		$f$に含まれるすべての単項式が、それぞれいずれかの$u \in M_n$で割り切れることが、$f$が$I$の元であるための必要十分条件である。
	\end{block}
\end{frame}

\begin{frame} {単項式イデアルの記述問題}
	単項式イデアルの記述問題(単項式イデアルが与えられたとき、それを有限生成なイデアルで記述できるかどうか)を考える。\\
	任意の単項式イデアル$I$は補題\ref{lem:monomial_ideal}の条件3から、$I$の単項式の線形結合で表せる(でもこれは、単項式イデアルの定義のような気もする)。\\
	したがって、$I=\langle \{u | u \in M \subset M_n \} \rangle$と書けるが、
	$\{u | u \in M \subset M_n \}$は有限集合とは限らないので、
	\red{生成系として不要な元}を取り除くこと(かつ同じものを生成するイデアル)を考える。\\
	\begin{exampleblock} {不要な元の例}
		不要な元の例として、以下のようなものが考えられる:
		\begin{itemize}
			\item $I=<xy, xy^2>$を考えると、$xy^2$は、$y \times (xy)$なので不要。こういった不要な元を取り除くことを考えていく。
			\item ある単項式$u \in I$は、それを割り切る単項式$v \in I$があれば、生成系には不要な元になる(補題\ref{lem:single_monomial}。
		\end{itemize}
	\end{exampleblock}
\end{frame}

\begin{frame} {極小元による生成系}
	取り除けない元を極小元として次に定義する。
	\begin{definition} {極小元}
		単項式の集合$M_n$の空でない部分集合$M$に対して、$u \in M$が$M$の極小元であるとは、
		任意の$v \in M$について、条件「$v$が$u$を割り切るならば$v=u$である」が成立することをいう。
	\end{definition}
	\begin{exampleblock} {極小元の例}
		\begin{itemize}
			\item $M = \{xy, xy^2, xy^3\}$の時、$u=xy$は、
			      $v=xy^2, xy^3$に割り切られないが、$v=xy$に割り切られていて、
									$v=u$が成り立っているので、$u=xy$は$M$の極小元
			\item $M = \{x^2y^2, x^5, x^10y^10, x^4y^4\}$の時、$u=x^5$は、
			他の元に割り切られていないので、極小元である。
			      $u=x^2y^2$も他に割り切られてないので、極小元である。
		\end{itemize}
	\end{exampleblock}
\end{frame}

\begin{frame}
    \begin{block} {コメント}
		この資料を準備していて、単項式の集合$M$における極小元というのは、$M$における素数のようなもの(単位元がないので違うが)だと感じた。
	\end{block}
	不要な元を削っていった結果、極小元が得られるが、それによって元の単項式イデアルを表現できることを次の補題で示す。
	\begin{lemma}
		単項式イデアル$I=\langle \{ u | u \in M \subset M_n \}\rangle$について、
		$M$のすべての極小元からなる集合は$I$の生成系となる
	\end{lemma}
	証明は次ページで行うが、この補題によって極小元によって単項イデアルは表現できる事がわかる(ただし、極小元の数は有限かどうかはこの補題では分からない)。
\end{frame}

\begin{frame}
	\begin{proof}
	    \only<1> {
					$M$の全ての極小元からなる集合を$\tilde{M}$とおき、
					$I_0 = \langle \{u | u \in \tilde{M} \} \rangle$とおく。
					このとき、$\tilde{M} \subset M$より、$I_0 \subset I$は自明である。\\
					$I \subset I_0$を示すため、$f \in I$が$I_0$に含まれることを示す。生成系の定義より、
					\begin{align*}
						f = \sum_{u \in M} g_u u, g_u \in K[x_1, \ldots, x_n]
					\end{align*}
					と書ける。$g_u$は有限個を除いて0。
					$g_u \neq 0$となる$u$について、
					$u$を割り切る単項式を$\tilde{M}$から選ぶことができるので、これを$v_u$とする。
					$h_u = g_u u / v_u$とすると、これは$K[x_1, \ldots, x_n]$の元であり、
					\begin{align*}
						f = \sum_{u \in M} h_u v_u
					\end{align*}
					と書くことができる。
		}
		\only<2> [証明続き] {
		$v_u \in \tilde{M}$についてまとめたもの(同じ$v_u$を与える$h_u$は和をとる)で改めて書き直すと、
		\begin{align*}
			f = \sum_{v \in \tilde{M}} h^*_v v
		\end{align*}
		$\{h^*_v | v \in M\}$も有限個を除いて0なので、$f \in I_0$である。
		}
	\end{proof}
\end{frame}

\begin{frame}
\begin{exampleblock} {極小元の生成系の例}
単項式イデアル$I = \langle xy^5, x^4y3, x^6 \rangle$は極小元は、$\{xy^5, x^4y^3, x^6\}$である。
この極小元に対して、冗長な元を適当に追加すると、例えば
\begin{align*}
    & xy^5, xy^6, x^2y^6, \\
    & x^5y^5, x^6, x^6y^3, \\
    & x^7y^4, x^7y^5, x^9
\end{align*}
を考えて、これらの生成系を作ることもできるが、この生成系は、極小元による生成系によって表現することができて、証明と同じように極小元にまとめる処理を行うと、
\begin{align*}
    & f = h_1 xy^5 + h_2 xy^6 + h_3 x^2y^6 + h_4 x^5y^5 + h_5 x^6 + h_6 x^6y^3 + h_7 x^7y^4 + h_8 x^7 y^5 + h_9 x^9\\
    & = (h_1 + yh_2 + xy h_3 + x^4 h_4) xy^5 + (h_5 + x^3 h_9) x^6 + (x^2 h_6 + x^3 y h_7 + x^3y^2 h_8) x^4y^3
\end{align*}
となる。なお、\red{表し方は一意でなく}て、後述の順序関係を与えれば、一意な構成方法を与えることは可能になる(はず)。
\end{exampleblock}
\end{frame}

\begin{frame} {Dicksonの補題}
ここまでで、単項式イデアルのイデアル記述問題(イデアルが有限生成である)は、単項式の任意の集合の極小元が高々有限個であることを示せれば良い事がわかった。
これは次のDicksonの補題によって保証される:
\begin{Theorem} [Dicksonの補題]
空でない単項式の任意の集合$M \subset M_n$の極小元は高々有限個である。
\end{Theorem}
\end{frame}
% \begin{frame}
% \end{frame}
